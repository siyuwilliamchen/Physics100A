\documentclass{article}
\usepackage[english]{babel}
\usepackage{amsmath}
\usepackage{amsfonts}
\usepackage{amsthm}
\usepackage{amssymb}
\usepackage{physics}

\usepackage{mathtools}

\newcommand\perm[2][^n]{\prescript{#1\mkern-2.5mu}{}P_{#2}}
\newcommand\comb[2][^n]{\prescript{#1\mkern-0.5mu}{}C_{#2}}

\DeclareMathOperator{\spn}{span}

\makeatletter
\renewcommand*\env@matrix[1][*\c@MaxMatrixCols c]{%
  \hskip -\arraycolsep
  \let\@ifnextchar\new@ifnextchar
  \array{#1}}
\makeatother

\newtheorem{theorem}{Theorem}[section]
\newtheorem{corollary}{Corollary}[theorem]
\newtheorem{lemma}[theorem]{Lemma}

\title{Problem Presentation \#3 Physics 100A}
\date{7/14/2023}
\author{William Chen 3619053} 
\begin{document}

\maketitle

\paragraph{Section 10, page 147: problem 7}
Show that, in $n$-dimensional space, any $n + 1$ vectors are linearly dependent.


\paragraph{(i) Introduction} 

This problem is pretty straightforward. We have to show that $n+1$ vectors in an $n$ dimensional vector space have to be linearly dependent. However, proving it without using the properties of a basis requires a little bit more thought. I chose to present this problem because I think the proof that I came up with was quite interesting.

We're only going to use the following theorem relating basis and vector space without proving them: All finite dimensional vectors spaces have basis with amount of vectors equal to their dimension.

\paragraph{(ii) Relationship to course material.}

The concept of linear independence is quite crucial to this course. It is fundamental to describe a lot of other properties of a vector field, such as dimensions. Since we discuss concepts like eigenvectors and eigenvalues quite often, this is a good basic concept to be familiar with.

\paragraph{(iii) Steps to solve the problem}

\begin{proof}
Suuppose we have a list of vectors $S = \{\vec{v}_1, \vec{v}_2 \ldots \vec{v}_n, \vec{v}_{n+1} \} \in V$, where $V$ is a $n$-dimensional space and the list $S$ is independent. Since $V$ is a finite dimensional vector space, it must have a basis with a list whose cardinality is equal to the dimension of the vector space. Let $B = \{ \vec{b}_1, \vec{b}_2 ,\ldots \vec{b}_n \}$ be a basis in $V$.

Then, let's consider since $\spn(B) = V, \vec{v}_1 \in V \Rightarrow \vec{v}_1 \in \spn(B), $

This can be written as 

\begin{align*}
\vec{v}_1 = \sum_{k=1}^{n} a_k \vec{b}_k
\end{align*}

Since the vector $\vec{v}_1$ is not zero, otherwise the list $S$ would not be independent, then at least one of the $a_k$ has to be non-zero. WLOG, we can call that term $a_k \vec{b}_k$ as $a_1 \vec{b}_1$ as the basis is not ordered. Then we can manipulate the equation algebraically to obtain

\begin{align*}
\vec{b}_1 = \frac{1}{a_1}(\vec{v}_1 - \sum_{k=2}^{n} a_k \vec{b}_k)
\end{align*}

Now $\vec{b}_1$ is in the linear combination of $\vec{v}_1$ and the rest of the basis. This is our base case that $\spn{\{\vec{v}_1, \vec{b}_2 ,\ldots , \vec{b}_n\}} = V$

To continue, we have to show that assume that $\spn{\{ \vec{v}_1 \ldots, \vec{v}_i , \vec{b}_{i+1}, \ldots, \vec{b}_n \}}=V$ is true and see that it implies $\spn{\{ \vec{v}_1 \ldots, \vec{v}_{i+1} , \vec{b}_{i+2}, \ldots, \vec{b}_n \}}=V$.

Since $\spn{\{ \vec{v}_1 \ldots, \vec{v}_i , \vec{b}_{i+1}, \ldots, \vec{b}_n \}}=V$, consider $\vec{v}_{i+1}$, we can write it as

\begin{align*}
\vec{v}_{i+1} = c_1 \vec{v}_1 + \ldots + c_i \vec{v}_{i} + \sum_{k=i+1}^{n} a_k \vec{b}_k
\end{align*}

but still, at least one constant in the remaining $a_k$ must be zero, otherwise we would violate the independence of vectors $\vec{v}$ as $\vec{v}_{i+1}$ is written as a linear combination of all of the vectors before it. So then, we can do the same process and WLOG name a non-zero $a_k \vec{b}_k$ term as $a_{i+2} \vec{b}_{i+2}$.

\begin{align*}
\vec{b}_{i+2} = \frac{1}{a_{i+2}}(\vec{v}_{i+2} - c_{i+1} \vec{v}_{i+1} - \ldots - c_1 \vec{v}_1 - \sum_{k=i+3}^{n} a_k \vec{b}_k)
\end{align*}

which shows that indeed, $\vec{b}_{i+2}$ is in the span of all of the vectors in $S$ up to $\vec{v}_{i+2}$ and the remainder of the basis, therefore, this implies $\spn{\{ \vec{v}_1 \ldots, \vec{v}_{i+1} , \vec{b}_{i+2}, \ldots, \vec{b}_n \}}=V$

Then, we can use induction and we have proved that the following is true:

\begin{align*}
\spn{\{ \vec{v}_1 \ldots, \vec{v}_{n} \}} = V
\end{align*}

Since for any independent list of $n+1$ elements, the first $n$ elements must hold true to the above, then we can say that $\vec{v}_{n+1} \in V \Rightarrow \vec{v}_{n+1} \in \spn{\{ \vec{v}_1 \ldots, \vec{v}_{n} \}} \Rightarrow \vec{v}_{n+1}$ is not linearly independent to $\{ \vec{v}_1 \ldots, \vec{v}_{n} \}$. Therefore we have a contradiction, and indeed such list of $11$ vectors cannot be independent, and therefore must be dependent.


\end{proof}

\paragraph{(iv) Solution Summary}

The proof is tedious, but does not involve any other property of the basis other than the property that a finite dimensional vector space must have a basis. It uses a combination of algebra to show that any 10 independent vectors would span the entire space of $V$, and we show that by borrowing an existing basis and substitute in elements from our vectors one by one until we have shown that our 10 vectors also span $V$. Finally, since the 10 vectors span $V$, and the 11 vectors is in $V$, they cannot be linearly independent.

\paragraph{(v) Notes on method}
Initially approaching this problem, I considered using a simple proof stating that, since a independent list of vectors has the same cardinality as the dimension of the space, then the list must be spanning and therefore a basis, but it just felt like circular reasoning since we could probably deduct that result using this, more appropriately. Then I ran into problems of actually coming up with a solution without using properties of a basis other than its spanning and independent definition, and then I came up with this. There is definitely simpler ways of doing this, but I like this one because it feels quite unique.


\end{document}