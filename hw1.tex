\documentclass{article}
\usepackage[english]{babel}
\usepackage{amsmath}
\usepackage{amsfonts}
\usepackage{amsthm}
\usepackage{amssymb}
\usepackage{physics}

\usepackage{mathtools}

\newcommand\perm[2][^n]{\prescript{#1\mkern-2.5mu}{}P_{#2}}
\newcommand\comb[2][^n]{\prescript{#1\mkern-0.5mu}{}C_{#2}}

\DeclareMathOperator{\spn}{span}

\makeatletter
\renewcommand*\env@matrix[1][*\c@MaxMatrixCols c]{%
  \hskip -\arraycolsep
  \let\@ifnextchar\new@ifnextchar
  \array{#1}}
\makeatother

\newtheorem{theorem}{Theorem}[section]
\newtheorem{corollary}{Corollary}[theorem]
\newtheorem{lemma}[theorem]{Lemma}

\title{HW\#2 Physics 100A}
\date{6/28/2023}
\author{William Chen 3619053} 
\begin{document}

\maketitle
 
\section*{Boas Chapter 3, Section 7, Pages 130} 
 
Are the following operators linear? 
 
\paragraph{Problems 9}  Find the square; operate on numbers or on functions.

\begin{proof}
First, the condition for operators $A$ from $V \to V$ to be linear is

\begin{align*}
A \ket{ \alpha \vec{v} + \lambda \vec{w}} = \alpha A \ket{\vec{v}} + \lambda A \ket{ \vec{w}}
\end{align*}

For the case on numbers, let $P : \mathbb{F} \to \mathbb{F} : P(a) = a^2$

\begin{align*}
P(ax + by) &= (ax + by)^2 = a^2 x^2 + 2axby + b^2 y^2 \\
aP(x) + bP(y) &= ax^2 + by^2
\end{align*}

Therefore LHS $\neq$ RHS, $P$ is not a linear operator.

For the case on functions, the square of the function can be defined to the composition of the function with itself. The square of a function is only defined if the co-domain is also the domain, in other words, when the function is an operator.

Let $P: V \to V: P(f) = f \circ f$ where $V$ is the set of all operators of set $A$.

\begin{align*}
P(\alpha f + \lambda g) &= (alpha f + \lambda g) \circ (alpha f + \lambda g) 
\\ &= \alpha^2 f \circ f + \alpha \lambda f \circ g + \alpha \lambda g \circ f + \lambda^2 g \circ g\\
\alpha P(f) + \lambda P(g) &= \alpha f \circ f + \lambda g \circ g
\end{align*}
 
Therefore LHS $\neq$ RHS, $P$ is not a linear operator.

Therefore finding square is not a linear operator for numbers and functions.

\end{proof}

\paragraph{Problem 13}

Let $W = \{f : \mathbb{C} \to \mathbb{C} \}$, let $D = \{W \to W : Df = \frac{d}{dx} f(x)\}$

(a) As in Problem 12, is $D^2 + 2D + 1$ linear?
\begin{proof}
Let $P : W \to W : P = D^2 + 2D + 1$.

\begin{align*}
P (\alpha f + \lambda g) &= \frac{d^2}{dx^2} (\alpha f + \lambda g) + \frac{d}{dx}  (\alpha f + \lambda g) +  \alpha f + \lambda g
\\ &= \alpha \frac{d^2}{dx^2} f + 2 \alpha \frac{d}{dx} f + \alpha f + \lambda \frac{d^2}{dx^2} g + 2 \lambda \frac{d}{dx} g + \alpha f + \lambda g
\\ \alpha P(f) + \lambda P(g)& = \alpha (\frac{d^2}{dx^2} f + \frac{d}{dx} f + f) + \lambda (\frac{d^2}{dx^2} g + \frac{d}{dx} g + g)
\end{align*}

Therefore LHS = RHS, $D^2 + 2D + 1$ is linear.
\end{proof}

(b) Is $x^2 D^2 - 2xD + 7$ a linear operator?

\begin{proof}

Let $P : W \to W : P = x^2 D^2 - 2xD + 7$

\begin{align*}
P(\alpha f + \lambda g) &= x^2 \frac{d^2}{dx} (\alpha f + \lambda g) - 2x (\alpha f + \lambda g) + 7(\alpha f + \lambda g) \\
\alpha P(f) + \lambda P(g) &= \alpha (x^2 \frac{d^2}{dx^2} f - 2x \frac{d}{dx} f + 7f) + \lambda (x^2 \frac{d^2}{dx^2} g - 2x \frac{d}{dx} g + 7g)
\end{align*}

Expanding, we would obtain that LHS = RHS, therefore $x^2 D^2 - 2xD + 7$ is linear.

\end{proof}

\paragraph{Problem 14}

\begin{proof}
Find the maximum; operate on functions of x.

Let $W = \{f : [a, b] \times i[c, d] \to \mathbb{C} \}$, Let $P : W \to W : W(f) = f_{max}$

Consider $f, g \in W$ such that

\begin{align*}
f(x) = \begin{cases}
1 (x = 3)\\
0 (x \neq 3)
\end{cases}
g(x) = \begin{cases}
1 (x = 1)\\
0 (x \neq 1)
\end{cases}
\end{align*}

The maximum value for $(\alpha f + \lambda g)$ depends on whether $\alpha$ is greater than $\lambda$, as the greater value would be the maximum value over the interval as every other value is 0.

\begin{align*}
P(\alpha f + \lambda g) &= \max{(\alpha, \lambda)} \\
\alpha P(f) + \lambda P(g) &= \alpha + \lambda
\end{align*}

Therefore LHS $\neq$ RHS, the maximum operator is not linear.
\end{proof}

\paragraph{Problem 16}
\begin{proof}

Find the inverse; operate on square matrices.

Let $P$ be an operator on all invertible square matrices such that $P(A) = A^{-1}$ where $A^{-1} A = I$.

\begin{align*}
P(\alpha A + \beta B) &= (\alpha A + \beta B)^{-1} \\
\alpha P(A) + \beta P(B) &= \alpha A^{-1} + \beta B^{-1}
\end{align*}

The two terms are not equal in general. A quick counter example is if $A = I, B = I$, following, $(\alpha A + \beta B)$ would have diagonal entries of $\alpha + \beta$, therefore an elementary matrix of scaling by $\frac{1}{\alpha + \beta}$ would be its inverse. For the second case, since $A^{-1} = I, B^{-1} = I$, $\alpha A^{-1} + \beta B^{-1}$ is a matrix of diagonal entries of $\alpha + \beta$, which is clearly different than the first case. Therefore the inverse operator for square matrices is not linear.

\end{proof}
\section*{Section 14, Page 184}

For each of the following sets, either verify (as in Example 1) that it is a vector space,
or show which requirements are not satisfied. If it is a vector space, find a basis and the
dimension of the space.

\paragraph{Problem 2}
Linear combinations of the set of functions ${e^x, \sinh x, xe^x}$.

\paragraph{Problem 8}
Polynomials of degree $\leq 7$ but with all odd powers missing.

\end{document}