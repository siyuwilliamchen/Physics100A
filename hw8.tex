\documentclass{article}
\usepackage{graphicx} % Required for inserting images
\usepackage[english]{babel}
\usepackage{amsmath}
\usepackage{amsfonts}
\usepackage{amsthm}
\usepackage{amssymb}
\usepackage{physics}
\numberwithin{equation}{section}


\usepackage{mathtools}

\newcommand{\Set}[1]{\{#1\}}
\newcommand{\FT}{\mathcal{F}}
\newcommand\perm[2][^n]{\prescript{#1\mkern-2.5mu}{}P_{#2}}
\newcommand\comb[2][^n]{\prescript{#1\mkern-0.5mu}{}C_{#2}}

\DeclareMathOperator{\spn}{span}

\makeatletter
\renewcommand*\env@matrix[1][*\c@MaxMatrixCols c]{%
  \hskip -\arraycolsep
  \let\@ifnextchar\new@ifnextchar
  \array{#1}}
\makeatother

\newtheorem{theorem}{Theorem}[section]
\newtheorem{corollary}{Corollary}[theorem]
\newtheorem{lemma}[theorem]{Lemma}
\title{HW\#8 PHYS100A}
\author{Siyu Chen}
\date{July 2023}

\begin{document}

\maketitle

\section{Boas Chapter 12: Section 1, Page 564: problem 5}
Solve the following differential equations by series and also by an elementary method and verify that your solutions agree. \begin{align*}
    y'' = y \end{align*}

Let's asssume that the solution to our differential equation can be written as a power series, we have:

\begin{align}
    y = \sum_{n=0}^{\infty} a_n x^n
\end{align}

Since our equation is $y'' = y$, let's compute $y''$

\begin{align}
    y'' = \sum_{n=0}^{\infty} \frac{d^2}{dx^2} a_n x^n = \sum_{n=2}^{\infty}n(n-1)a_n x^{n-2}
\end{align}

Equating the two, we have 

\begin{align}
    a_0 + a_1 x + a_2 x^2 + a_3 x^3 + \ldots = 2 \cdot 1 a_2 + 3 \cdot 2 a_3 x + 4 \cdot 3 a_4 x^2 + 5 \cdot 4 a_5^3 + \ldots
\end{align}

relating the $a_n$ constants, we have:

\begin{align}
    a_n = n(n-1) a_{n+2} 
\end{align}

Let $n=2m$, the even and odd terms can be represented respectively as: 

\begin{align}
    y_{even} &= a_0 + a_0\frac{x^2}{2!} + a_0  \frac{x^3}{4!} = a_0 \sum_{m=0}^{\infty} \frac{x^{2m}}{2m!} \\
    y_{odd}  &= a_1 \sum \frac{x^{2m+1}}{(2m+1)!}
\end{align}

By inspection, we can see that the even and odd series correspond to the $\cosh(x), \sinh(x)$ function, 

\begin{align}
    y &= y_{even} + y_{odd} = \frac{a_0}{2} (e^x + e^{-x}) + \frac{a_1}{2} (e^x - e^{-x}) \\
    y &= \frac{a_0 + a_1}{2} e^x + \frac{a_0- a_1}{2} e^{-x}
\end{align}
Let $c_1 = \frac{a_0 + a_1}{2}, \frac{a_0- a_1}{2}$, we have 

\begin{align}
    y = c_1 e^x + c_2 e^{-x}
\end{align}

Doing it by an elementary method, we can arrange it in the form of 

\begin{align}
    y'' - y = 0
\end{align}

The characteristics equation of this differential equation is \begin{align}
    \lambda^2 - 1 = 0
\end{align}
yielding $\lambda = 1, -1$, so we can say that the solution of this homogenous linear 2nd order ODE is of the form

\begin{align}
    y = c_1 e^{\lambda_1 x} + c_2 e^{\lambda_2 x} = c_1 e^x + c_2 e^{-x}
\end{align}

which agrees with our result from the series method.

\section{Additional Problem G1}

\textbf{Part (a):}

The operator $\hat{L}$ in this case has terms $\alpha(x) = 1, \beta(x) = 0, \gamma(x) = 0$, therefore it can be written as 

\begin{align}
    \hat{L} = \frac{d^2}{dx^2}
\end{align}

\textbf{Part (b):}

Since the interval on which our functions are defined is $[0,1]$, that is our limit of integration.

Our weight function is given by the following formula:

\begin{align}
    w(x) = \frac{1}{\alpha(x)} e^{\int^{x} \frac{\beta(x')}{\alpha(x')}dx'} = \frac{1}{1} e^0 = 1
\end{align}

So our integral form of our inner product is

\begin{align}
    \bra{f} \ket{g} = \int_{0}^{1} f^*(x) g(x) dx
\end{align}

\textbf{Part (c):}

To make sure that $\hat{L}$ is Hermitian, most generally, it must satisfy this:

\begin{align}
    & w(x) \alpha(x) (u^*_m (x) u'_n(x) - u_n(x) {u^*_m}^{'} (x))  \rvert_{0}^{1} = 0
\end{align}

since $u_n (0) = 0, u'_n(0) = 0$, we can show that

\begin{align}
    ( u^*_m (1) u'_n (1) - u_n (1) {u^*_m}^{'} (1))- ( u^*_m (0) u'_n (0) - u_n (0) {u^*_m}^{'} (0)) = 0
\end{align}

so indeed $\hat{L}$ is Hermitian.

\textbf{Part (d):}

Since the problem really describes the following differential equation with boundary condition of $u'(1) = 0, u(0) = 0$:

\begin{align}
    u'' - \lambda u = 0
\end{align}

(And the homework did not specify that we use any specific methods), we can simply treat this as a 2nd order homogeneous linear ODE, with characteristics equation

\begin{align}
    r^2 - \lambda = 0 
\end{align}

yielding results of $r_1 = \sqrt{\lambda}, r_2 = -\sqrt{\lambda}$

so our solution is given in the form

\begin{align}
    u(x) = c_1 e^{\sqrt{\lambda}x} + c_2 e^{-\sqrt{\lambda}x}
\end{align}

checking for the boundary condition, we have

\begin{align}
    u(0) &= c_1 + c_2 = 0 \Rightarrow c_1 = - c_2 \\
    u'(1) &= c (\sqrt{\lambda}e^{\sqrt{\lambda}} + \sqrt{\lambda}e^{-\sqrt{\lambda}}) = 0 \Rightarrow  \\
    &e^{\sqrt{\lambda}} + e^{-\sqrt{\lambda}} = 0
\end{align}

Taking (2.11), we can find our requirement for our $\lambda$, we inspect that (2.11) indicates

\begin{align}
    \cosh (\sqrt{\lambda}) = 0
\end{align}

we know that the zeros for $\cosh$ is given by the following by Euler's identity, the proof I will omit:

\begin{align}
    \cosh(x) = 0 \Rightarrow x = (n\pi + \frac{\pi}{2}) i, n \in \mathbb{Z}
\end{align}

then we have

\begin{align}
    \lambda_n = - (n^2 \pi^2 + n \pi^2 + \frac{\pi^2}{4})
\end{align}

and that is our eigenvalues $\lambda_n$. Moving onto finding our eigenfunctions, recall (2.8), along with the restriction of (2.9), we have 

\begin{align}
    u_n (x) = c_1 (e^{i(n\pi + \frac{\pi}{2})x} - e^{- i(n\pi + \frac{\pi}{2})x})= c \sin((n\pi + \frac{\pi}{2})x), c \in \mathbb{C}, n \in \mathbb{Z} 
\end{align} 

(Note, the $c_1$ and the $c$ are off by a factor of $2i$ from the $\sin$ expansion, so they are different, but they are complex constants, so it does not really matter.)

\section{Additional Problem G2}

We can identify the differential operator $\hat{L}$ as \begin{align}
    \hat{L} = \alpha(x) \frac{d^2}{dx^2} + \beta(x) \frac{d}{dx} + \gamma(x) 
\end{align}

where $\alpha(x) = 1-x^2, \beta (x) = -x, \gamma(x) = 0$. The limits of integration is the same as the interval for which the functions are defined, or $[-1,1]$, so our weight function has the form

\begin{align}
    w(x) &= \frac{1}{\alpha(x)} e^{\int^{x}\frac{\beta(x')}{\alpha(x')}dx'}
\end{align}

evaluate the integral in the exponent. Using u-substitution, $u = x^2-1$, the integral evaluates to $\ln(u)$, we have 

\begin{align}
    \int^{x}\frac{x'}{x'^2-1}dx' = \frac{1}{2} \ln \vert x^2-1 \vert
\end{align}

then, the weight function is 

\begin{align}
    w(x) = \frac{1}{1-x^2} e^{\frac{1}{2}\ln \vert x^2 - 1 \vert } = \frac{\vert x^2-1 \vert^\frac{1}{2}}{1-x^2} 
\end{align}

since $x \in [-1,1]$ from the definition of the problem, $1-x^2 \geq 0$, we can simply the expression to 

\begin{align}
    w(x) = \frac{1}{\sqrt{1-x^2}}
\end{align}

Now, verify that it is Hermitian, it has to once again satisfy the equivalent of (2.4). Verifying with the condition $f(-1) = f(1) = 0$, we have:
\begin{align}
    &w(x) \alpha(x) (f^*(x) g'(x) - g(x) f'^*(x)) \vert^{1}_{-1} = \frac{1-x^2}{\sqrt{1-x^2}} (f^*(x) g'(x) - g(x) f'^*(x)) \vert^{1}_{-1} \\ &= \sqrt{1-x^2} ((0 g'(1) - 0 f'^*(1))-(0 g'(-1) - 0 f'^*(-1)) = 0
\end{align}

\section{Additional Problem G3}

\textbf{Part (a):}

The inner product, with the defined weight function, is of the form: 

\begin{align}
    \bra{f} \ket{g} = \lim_{a\to \infty} \int_{-a}^{a} e^{-x^2} f^*(x) g(x) dx
\end{align}

And our operator is of the form 
\begin{align}
    \hat{L} = \alpha(x) \frac{d^2}{dx^2} + \beta(x) \frac{d}{dx} + \gamma(x) 
\end{align}

where $\alpha(x) = 1, \beta (x) = -2x, \gamma(x) = 0$.

To verify that the operator is Hermitian, we have to show this again: 

\begin{align}
   \lim_{a \to \infty} w(x) \alpha(x) (f^*(x) g'(x) - g(x) f'^*(x)) \vert^{a}_{-a} = 0
\end{align}

The expression becomes

\begin{align}
    \lim_{a \to \infty} e^{-a^2} ((f^*(a) g'(a) - g(a) f'^*(a))-(g'(-a) - g(-a) f'^*(-a)) 
\end{align}

Since we're not worrying about our function potentially diverging at the end points, and we can see that $\lim_{a \to \infty} e^{-a^2} = 0$, the whole expression is just $0$ multiplying some term that converges, therefore $0$. Therefore we verified that the operator is Hermitian.

\textbf{Part (b):}

From our previous homework, we have, for our orthogonal basis vectors: \begin{align}
    \ket{e_1} &= 1 \\
    \ket{e_2} &= x \\
    \ket{e_3} &= x^2 - \frac{1}{2}
\end{align}

Verifying $\ket{e_1}$, we have 

\begin{align}
    -\lambda_1 1 = 0 \Rightarrow \lambda_1 = 0
\end{align}

Verifying $\ket{e_2}$, we have

\begin{align}
    -2x - \lambda_2 x = 0 \Rightarrow x( 2 + \lambda_2) = 0 \Rightarrow \lambda_2 = -2
\end{align}

Verifying $\ket{e_3}$, we have

\begin{align}
    2 - 4x^2 - \lambda_3 (x^2 - \frac{1}{2}) = 0 \Rightarrow (x^2 - \frac{1}{2}) (-4 - \lambda_3) \Rightarrow \lambda_3 = -4
\end{align}

and we see that indeed they satisfy the differential equations with eigenvalues $0, -2, -4$ respectively.

\end{document}
